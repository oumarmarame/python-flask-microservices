\documentclass[11pt,a4paper]{article}
\usepackage[utf8]{inputenc}
\usepackage[T1]{fontenc}
\usepackage{geometry}
\usepackage{hyperref}
\usepackage{graphicx}
\usepackage{amsmath}
\geometry{margin=2.5cm}
\title{TP 1 -- Mise en \OE uvre d'un Pipeline de Journalisation, Traçage et Métriques avec OpenTelemetry}
\author{Oumar Marame}
\date{26 octobre 2025}

\begin{document}
\maketitle
\begin{center}
\textbf{MGL870 -- Observabilité des systèmes logiciels} \\
\textit{École de Technologie Supérieure (ÉTS)} \\
Professeur du cours MGL870
\end{center}

\begin{abstract}
Ce rapport présente la mise en œuvre complète d'un pipeline d'observabilité pour une application microservices e-commerce Python/Flask. Le projet intègre OpenTelemetry pour l'instrumentation, avec collecte de traces (Jaeger), métriques (Prometheus) et logs (Loki), le tout visualisé dans Grafana. L'architecture déployée via Docker Compose comprend 4 microservices instrumentés, un OpenTelemetry Collector centralisé, et un système d'alerting Prometheus/Alertmanager. Le rapport documente également l'implémentation de spans personnalisés, de métriques métier, et une évaluation de maturité atteignant le Niveau 3 (Proactive) du modèle d'observabilité.
\end{abstract}

\tableofcontents
\newpage